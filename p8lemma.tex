\documentclass{article}
\usepackage[english]{babel}
\usepackage{amsthm}

\newtheorem{theorem}{Theorem}[section]
\newtheorem{corollary}{Corollary}[theorem]
\newtheorem{lemma}[theorem]{Lemma}

\theoremstyle{definition}
\newtheorem{definition}{Definition}[section]

\begin{document}
\section{Numbered theorems, definitions, corollaries and lemmas}
Theorems can easily be defined:

\begin{theorem}
Let \(f\) be a function whose derivative exists in every point, then \(f\) is 
a continuous function.
\end{theorem}

\begin{theorem}[Pythagorean theorem]
\label{pythagorean}
This is a theorem about right triangles and can be summarised in the next 
equation 
\[ x^2 + y^2 = z^2 \]
\end{theorem}

And a consequence of theorem \ref{pythagorean} is the statement in the next 
corollary.

\begin{corollary}
There's no right rectangle whose sides measure 3cm, 4cm, and 6cm.
\end{corollary}

You can reference theorems such as \ref{pythagorean} when a label is assigned.

\begin{lemma}
Given two line segments whose lengths are \(a\) and \(b\) respectively there is a 
real number \(r\) such that \(b=ra\).
\end{lemma}

\begin{definition}[Absolute value function] 
The absolute value function can be specified as a two-part definition as follows: \\
$
|x| =
\left\{
	\begin{array}{ll}
		x  & \mbox{if } x \geq 0 \\
		-x & \mbox{if } x < 0
	\end{array}
\right.
$
\end{definition}

\end{document}
